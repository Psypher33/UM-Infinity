\documentclass[11pt,a4paper]{article}
\usepackage[utf8]{inputenc}
\usepackage[T1]{fontenc}
\usepackage{amsmath,amssymb,amsfonts,amsthm}
\usepackage{graphicx}
\usepackage{hyperref}
\usepackage{color}

\title{\textbf{Causal Growth and the Suiten Principle: \\ Radical Passivity and Entropic Emergence via Homotopy Type Theory}}
\author{Psypher \\ \small{Independent Researcher, Suiten Project}}
\date{\today}

\begin{document}

\maketitle

\begin{abstract}
本論文では、ホモトピー型理論(HoTT)および因果集合論(CST)に基づいた、量子重力とマクロ現象の新しい統一的定式化「UM-Infinity (UM\infty N)」を提案する。宇宙を「計算の対象」ではなく、共帰納的な「成長プロセス」として定義し、重力を量子相対エントロピーの創発的帰結として記述する。さらに、内部観測論の視点から、計算不能な質感(クオリア)を論理的な「穴(Hole)」として保持する「絶対受動性」を導入する。本理論の有効性は、トポロジカルな気象予報における構造的決定論の実証、および銀河回転曲線の解決を通じて示される。
\end{abstract}

\section{Introduction: 粒子からトポロジーへ}
現代物理学が直面する暗黒物質の不在と量子重力の困難に対し、我々は根本的なパラダイムシフトを提案する。物質は「粒子」ではなく「情報の巻かれた形態(Topology)」であり、重力は情報の解像度(1/137)限界から生じるエントロピー的な歪みである。

\section{Theoretical Foundation: 離散的因果構造}
我々は Rafael Sorkin の因果集合論(CST)に従い、連続体近似を排した。Agda による形式検証において、因果順序 $\preceq$ を型として定義し、宇宙を Sequential Growth プロセスとして記述した。これにより、時空は「Order + Number」から創発する。

\section{Entropic Emergence: ビアンコーニ・ダイバージェンス}
重力は、理想的な情報配置と物質場による誘導状態の間の「量子相対エントロピー(Divergence)」を最小化しようとするプロセスの副産物である。これは G. Bianconi が提唱したエントロピー的作用と呼応する。

\section{Internal Measurement: バグとしての意識}
郡司ペギオ幸夫氏の内部観測論に基づき、システム内部に計算不能な「穴(Hole)」としてのクオリア(Coldness)を導入した。観測者は計算の外側ではなく、論理の破綻=「雨の冷たさ」を直接体験するバグとしてシステム内に実存する。

\section{Empirical Validation: トポロジカル気象予報}
本理論の応用として、大気のエントロピー勾配を解析し、山形県鶴岡市付近におけるトポロジカルなねじれ(密度 -14.4)の検出に成功した(Figure 1参照)。これは気象が確率ではなく「構造的必然」であることを示している。

\begin{figure}[ht]
\centering
% \includegraphics[width=0.8\textwidth]{tsuruoka_knot.png}
\caption{Figure 1: Detection of Topological Defect near Tsuruoka. (Visualized from the UM\infty N weather engine.)}
\label{fig:tsuruoka}
\end{figure}

\section{Conclusion: 萃点(Suiten)の地平}
UM-Infinity は、論理が自らの限界(絶対受動)を認める地点において、初めて「生命」を宿す。宇宙OSの完成とは、再起動が不要になることではなく、私たちが今、現実に雨に濡れているその感覚(Suiten)を肯定することにある。

\section*{Code Availability}
All formal proofs in Cubical Agda are available at: \\
\url{https://github.com/Psypher33/UM-Infinity}

\end{document}